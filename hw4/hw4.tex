%\documentclass[UTF8]{ctexart} % use larger type; default would be 10pt
\documentclass[a4paper]{article}
\usepackage{xeCJK}
%\usepackage{ctex}
%\usepackage{luatexja-fontspec}
%\setmainjfont{FandolSong}
%\usepackage[utf8]{inputenc} % set input encoding (not needed with XeLaTeX)

%%% Examples of Article customizations
% These packages are optional, depending whether you want the features they provide.
% See the LaTeX Companion or other references for full information.

%%% PAGE DIMENSIONS
\usepackage{geometry} % to change the page dimensions
\geometry{a4paper} % or letterpaper (US) or a5paper or....
% \geometry{margin=2in} % for example, change the margins to 2 inches all round
% \geometry{landscape} % set up the page for landscape
%   read geometry.pdf for detailed page layout information

\usepackage{graphicx} % support the \includegraphics command and options
\usepackage{bbold}
\usepackage{bm}
% \usepackage[parfill]{parskip} % Activate to begin paragraphs with an empty line rather than an indent

%%% PACKAGES
\usepackage{booktabs} % for much better looking tables
\usepackage{array} % for better arrays (eg matrices) in maths
\usepackage{paralist} % very flexible & customisable lists (eg. enumerate/itemize, etc.)
\usepackage{verbatim} % adds environment for commenting out blocks of text & for better verbatim
\usepackage{subfig} % make it possible to include more than one captioned figure/table in a single float
% These packages are all incorporated in the memoir class to one degree or another...

%%% HEADERS & FOOTERS
\usepackage{fancyhdr} % This should be set AFTER setting up the page geometry
\pagestyle{fancy} % options: empty , plain , fancy
\renewcommand{\headrulewidth}{0pt} % customise the layout...
\lhead{}\chead{}\rhead{}
\lfoot{}\cfoot{\thepage}\rfoot{}

%%% SECTION TITLE APPEARANCE
\usepackage{sectsty}
\allsectionsfont{\sffamily\mdseries\upshape} % (See the fntguide.pdf for font help)
% (This matches ConTeXt defaults)

%%% ToC (table of contents) APPEARANCE
\usepackage[nottoc,notlof,notlot]{tocbibind} % Put the bibliography in the ToC
\usepackage[titles,subfigure]{tocloft} % Alter the style of the Table of Contents
\renewcommand{\cftsecfont}{\rmfamily\mdseries\upshape}
\renewcommand{\cftsecpagefont}{\rmfamily\mdseries\upshape} % No bold!

%%% END Article customizations

%%% The "real" document content comes below...

\setlength{\parindent}{0pt}
\usepackage{physics}
\usepackage{amsmath}
%\usepackage{symbols}
\usepackage{AMSFonts}
\usepackage{bm}
%\usepackage{eucal}
\usepackage{mathrsfs}
\usepackage{amssymb}
\usepackage{float}
\usepackage{multicol}
\usepackage{abstract}
\usepackage{empheq}
\usepackage{extarrows}
\usepackage{textcomp}
\usepackage{mhchem}
\usepackage{braket}
\usepackage{siunitx}
\usepackage[utf8]{inputenc}
\usepackage{tikz-feynman}
\usepackage{feynmp}


\DeclareMathOperator{\p}{\prime}
\DeclareMathOperator{\ti}{\times}

\DeclareMathOperator{\e}{\mathrm{e}}
\DeclareMathOperator{\I}{\mathrm{i}}
\DeclareMathOperator{\Arg}{\mathrm{Arg}}
\newcommand{\NA}{N_\mathrm{A}}
\newcommand{\kB}{k_\mathrm{B}}

\DeclareMathOperator{\ra}{\rightarrow}
\DeclareMathOperator{\llra}{\longleftrightarrow}
\DeclareMathOperator{\lra}{\longrightarrow}
\DeclareMathOperator{\dlra}{\;\Leftrightarrow\;}
\DeclareMathOperator{\dra}{\;\Rightarrow\;}

%%%%%%%%%%%% QUANTUM MECHANICS %%%%%%%%%%%%%%%%%%%%%%%%
\newcommand{\bkk}[1]{\Braket{#1|#1}}
\newcommand{\bk}[2]{\Braket{#1|#2}}
\newcommand{\bkev}[2]{\Braket{#2|#1|#2}}

\DeclareMathOperator{\na}{\bm{\nabla}}
\DeclareMathOperator{\nna}{\nabla^2}
\DeclareMathOperator{\drrr}{\dd[3]\vb{r}}

\DeclareMathOperator{\psis}{\psi^\ast}
\DeclareMathOperator{\Psis}{\Psi^\ast}
\DeclareMathOperator{\hi}{\hat{\vb{i}}}
\DeclareMathOperator{\hj}{\hat{\vb{j}}}
\DeclareMathOperator{\hk}{\hat{\vb{k}}}
\DeclareMathOperator{\hr}{\hat{\vb{r}}}
\DeclareMathOperator{\hT}{\hat{\vb{T}}}
\DeclareMathOperator{\hH}{\hat{H}}

\DeclareMathOperator{\hL}{\hat{\vb{L}}}
\DeclareMathOperator{\hp}{\hat{\vb{p}}}
\DeclareMathOperator{\hx}{\hat{\vb{x}}}
\DeclareMathOperator{\ha}{\hat{\vb{a}}}
\DeclareMathOperator{\hs}{\hat{\vb{s}}}
\DeclareMathOperator{\hS}{\hat{\vb{S}}}
\DeclareMathOperator{\hSigma}{\hat{\bm\Sigma}}
\DeclareMathOperator{\hJ}{\hat{\vb{J}}}

\DeclareMathOperator{\Tdv}{-\dfrac{\hbar^2}{2m}\dv[2]{x}}
\DeclareMathOperator{\Tna}{-\dfrac{\hbar^2}{2m}\nabla^2}

%\DeclareMathOperator{\s}{\sum_{n=1}^{\infty}}
\DeclareMathOperator{\intinf}{\int_0^\infty}
\DeclareMathOperator{\intdinf}{\int_{-\infty}^\infty}
%\DeclareMathOperator{\suminf}{\sum_{n=0}^\infty}
\DeclareMathOperator{\sumnzinf}{\sum_{n=0}^\infty}
\DeclareMathOperator{\sumnoinf}{\sum_{n=1}^\infty}
\DeclareMathOperator{\sumndinf}{\sum_{n=-\infty}^\infty}
\DeclareMathOperator{\sumizinf}{\sum_{i=0}^\infty}

%%%%%%%%%%%%%%%%% PARTICLE PHYSICS %%%%%%%%%%%%%%%%
\DeclareMathOperator{\hh}{\hat{h}}               % helicity
\DeclareMathOperator{\hP}{\hat{\vb{P}}}          % Parity
\DeclareMathOperator{\hU}{\hat{U}}
\DeclareMathOperator{\hG}{\hat{G}}

\DeclareMathOperator{\GeV}{\si{GeV}}
\DeclareMathOperator{\LI}{\mathscr{L}.I.}
%\DeclareMathOperator{\g5}{\gamma^5}
\DeclareMathOperator{\gmuu}{\gamma^\mu}
\DeclareMathOperator{\gmud}{\gamma_\mu}
\DeclareMathOperator{\gnuu}{\gamma^\nu}
\DeclareMathOperator{\gnud}{\gamma_\nu}

\renewcommand{\u}{\mathrm{u}}
\renewcommand{\d}{\mathrm{d}}
\DeclareMathOperator{\s}{\mathrm{s}}

\DeclareMathOperator{\q}{\mathrm{q}}
\DeclareMathOperator{\bq}{\bar{\mathrm{q}}}

\DeclareMathOperator{\g}{\mathrm{g}}
\DeclareMathOperator{\W}{\mathrm{W}}
\DeclareMathOperator{\Z}{\mathrm{Z}}

%%% Feynman Diagram
\newcommand{\pa}{particle}
\newcommand{\mo}{momentum}
\newcommand{\el}{edge label}

%%% MQC
\newcommand{\iden}{{\large \bm{1}}}
\newcommand{\qed}{$ \Square $}
\newcommand{\tPhi}{\tilde{\Phi} }
\newcommand{\hsP}{\hat{\mathscr{P}}}
\newcommand{\hsS}{\hat{\mathscr{S}}}

\newcommand{\dis}{\displaystyle}
\numberwithin{equation}{section}

\title{Homework 4}
%\date{} % Activate to display a given date or no date (if empty),
         % otherwise the current date is printed 

\begin{document}
% \boldmath
\maketitle

\section{The Exchange Energy of a Uniform Electron Gas}
The exchange energy term (自旋轨道, $r$包括自旋) is that:
\begin{equation}
	E_{\text{exc}}= -\frac{1}{2}\sum_{i}\sum_{j} \int \mathrm{d} \bm{r} \int \mathrm{d} \bm{r'}
	\frac{\phi_i^*(\bm{r})\phi_j^*(\bm{r'})\phi_i(\bm{r}')\phi_j(\bm{r})}{|\bm{r}-\bm{r'}|}
\end{equation}
If we take the interaction term as pertubation to the original system, so the base wave function of
the system is plane wave:
\begin{equation}
	\phi(\bm{r}) = \frac{1}{\sqrt{V}} \exp(i \bm{k} \bm{r})
\end{equation}
The exchange term can be write as (相同自旋才有交换作用,对自旋求和相当于乘2, 利用库伦势场-- 汤川势的傅里叶变换, 下面开始$r$不包括自旋):
\begin{equation}
	\begin{aligned}
		E_{\text{exc}} & = - \frac{1}{2V^2} \sum_{\sigma}\sum_{\bm{k}}\sum_{\bm{k}'}
		\int \mathrm{d} \bm{r} \, e^{i \bm{(k'-k)r}}\int \mathrm{d} \bm{r'}\,
		\frac{ e^{i \bm{(k-k')r'}}}{|\bm{r}-\bm{r'}|}                                                            \\
		               & =-\frac{1}{V^2} \sum_{\bm{k}}\sum_{\bm{k}'}
		\int \mathrm{d} \bm{r} \,4\pi e^{i \bm{(k'-k)r}}  \frac{e^{i(\bm{k}-\bm{k}')\bm{r}}}{|\bm{k}-\bm{k}'|^2} \\
		               & = - \frac{4\pi}{V} \sum_{\bm{k}}\sum_{\bm{k}'} \frac{1}{|\bm{k}-\bm{k}'|^2}
	\end{aligned}
\end{equation}
Under quasicontinuous approximation, we have that:
\begin{equation}
	\sum_{\bm{k}} \rightarrow \frac{V}{(2\pi)^3}\int_0^{k_f} k^2 \mathrm{d}k  \int_0^{\pi}\mathrm{d} \theta \sin \theta\int_0^{2\pi} \mathrm{d} \varphi
\end{equation}
Therefore,
\begin{equation}
	\begin{aligned}
		E_{\text{exc}} & = -\frac{1}{\pi}\sum_{\bm{k}}\int_0^{k_f} k'^2 \mathrm{d}k'  \int_0^{\pi} \mathrm{d} \theta \,\frac{\sin \theta }{k^2+k'^2+2kk'\cos\theta} \\
	\end{aligned}
\end{equation}
Take $k_f = (3\pi^2 \rho)^{1/3}$, then we can get $E_{\text{exc}}/N$. I just take the result from wiki:
\begin{equation}
	\frac{E_{\text{exc}}}{N} = - \frac{0.916}{\left(\frac{3}{4\pi n}\right)^{1/3}}
\end{equation}
\section{Correlated Quantum Chemistry Methods}

End-to-end Symmetry Preserving Inter-atomic Potential Energy Model for Finite and Extended
Systems.

The authors proposed a well-developed compuatational technique -- DeePMD-kit. Deep Potential is to 
employ deep learning techniques and 
realize an inter-atomic potential energy model that is general and accurate. 
The key component is to apply symmetry-invariant properties of a potential energy model by assigning a local reference frame and a local environment to each atom. 


They do the following experiment:
\begin{enumerate}
	\item The DeepPotential Model  for the small molecular system. 
	\item The DeepPotential Model for MoS2 and Pt system. 
	\item The DeepPotential Model  for Co, Cr, Fe, Mn, Ni, HEA system. 
	\item The DeepPotential Model for the TiO2 system, which contains 3 different polymorphs.
\end{enumerate}
	
This paper is quite interesting, because it combines symmetry in many body quantum systems and machine learning.
Because I am not an expert in machine learning, I cannot express the calculation in detail. If you are interested
in this paper, just download and run the code.

$$R_{ij}= \sum_{m,n} \braket{m,n|\sigma_i m, \sigma_j, n}$$
\end{document}